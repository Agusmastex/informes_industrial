\documentclass[a4paper,12pt]{article} % This defines the style of your paper
\usepackage{graphicx} 
% The default setting of LaTeX is to indent new paragraphs. This is useful for articles. But not really nice for homework problem sets. The following command sets the indent to 0.
\usepackage{setspace}
\setlength{\parindent}{0in}
% Package to place figures where you want them.
\usepackage{float}
\usepackage{multicol}
\usepackage{nccmath}
\usepackage{mathrsfs}
\usepackage{amssymb} %simbolos matematicos
\usepackage{ulem} 
\usepackage{cancel} %tachar con flecha
\usepackage{color,soul} %resaltador
\usepackage{lipsum}
% The fancyhdr package let's us create nice headers.
\usepackage{fancyhdr}
\usepackage[
backend=biber,
style=numeric,
sorting=nty
]{biblatex}
\addbibresource{sample.bib} %Imports bibliography file
\title{Bibliography management: \texttt{biblatex} package}

% 3. Header (and Footer)
\pagestyle{fancy} % With this command we can customize the header style.
\fancyhf{} % This makes sure we do not have other information in our header or footer.
\lhead{\footnotesize FCQ-UNA}
\rhead{\footnotesize Laboratorio de Analisis Industrial - Practica N°2 } 
\cfoot{\footnotesize \thepage} 

% 4. Your document
\begin{document}

% Title section of the document
\thispagestyle{empty} 
\begin{tabular}{p{15.5cm}}
{\large \bf Universidad Nacional de Asunción} \\
Facultad de Ciencias Químicas \\ 
Ingeniería Química - Laboratorio de Análisis Industrial \\
\hline
\end{tabular} 

\vspace*{0.3cm} % Vertical space between the line and title.

\begin{center} 
    {\Large \bf Practica N°2  \\ \vspace{3mm} Análisis de azúcar}
    \vspace{5mm}
    \begin{figure}[H] 
	    \centering
        \includegraphics{LOGO-UNA.jpg}
    \end{figure}
\end{center}

\vspace{15mm}

\begin{itemize}
    \item \textbf{Grupo 11: Integrantes}
    \begin{itemize}
        \item{\bf Mateo Augusto Acevedo Onieva}
        \item{\bf Dylan Sebastián Galeano Monteggia}
        \item{\bf José Manuel Karjallo Zárate}
    \end{itemize}
\end{itemize}



\newpage

\section{Determinación de azucares reductores}

\subsection{Fundamento} 
Se basa en la propiedad que tiene la sacarosa de invertirse (mediante hidrólisis ácida y térmica), lo cual permite la cuantificación de azucares reductores.

\subsection{Materiales} 
\begin{itemize}
\item {Baño de agua}
\item {Bureta de 50 cm3 graduada en 0.1 cm3}
\item {Fuente de calor con regulador de temperatura}
\item {Matraces aforados de 100, 200 y 1000 cm3}
\item {Matraz Erlenmeyer de 300 cm3}
\item {Pipetas volumétricas de 5 y 10 mL}
\item {Material común de laboratorio}
\item {Termómetro con escala de 0 a 100°C}
\item {Filtro ayuda}
\item {Papel filtro para azúcar}
\item {Balanza analítica}
\end{itemize}

\subsection{Reactivos}
\begin{itemize}
\item {Solución de Fehling Causse-Bonnans}
\item {Solución de azúcar invertido en agua: Disolver 10 g de glucosa y llevarlos a 1000 cm3}
\item {Solución indicadora de azul de metileno: Disolver 1 g de azul de metileno en agua y llevar a 100 cm3}
\item {Oxalato de sodio seco}
\end{itemize}

\subsection{Datos experimentales}
\textbf{Muestra:} Azúcar orgánica\\
\textbf{Color: Marrón} \\
\textbf{Estado físico:} Sólido \\ 
$V_c$ : Volumen consumido de solución de azúcar en la titulación 
$V_c = 9,4$ mL \\ 
\textbf{Observación:} Cambio de color en la solución a un color amarillo seguido de un rápido oscurecimiento 

\subsection{Discusión}
Se realizo la practica de determinación de azucares reductores de manera cualitativa, esto se debe a que durante la preparación de la solución de azúcar invertido se derramo una parte de la solución antes de llevar la solución a un volumen exacto, por lo que se decidió continuar la practica de manera cualitativa. Se logro observar un cambio de color a amarillo al consumir 9,4 ml de la muestra preparada, indicando así el punto final de la titulación, a diferencia del color rojo ladrillo esperado establecido en las normas y bibliografías consultadas, esto puede deberse a algunos factores, la concentración del azúcar invertido que se preparo, otros compuestos presentes en la muestra de azúcar utilizada, la utilización del reactivo de Fehling Causse-Bonnans a diferencia de la preparación de los reactivos de Fehling A y B provistos en la guía y la norma NMX-F-496-SCFI-2011 - DETERMINACIÓN DE REDUCTORES TOTALES EN AZUCARES Y MATERIALES AZUCARADOS



\newpage

\section{Cenizas sulfatadas}

\subsection{Fundamento}
Se basa en una incineración simple a 600 °C. La sulfatación del azúcar antes de la ignición ayuda a obtener cenizas libres de carbón a la temperatura baja de ignición empleada

\subsection{Materiales} 
\begin{itemize}
    \item{Placa calefactora }
    \item{Espátula metálica}
    \item{Desecador con agente desecante}
    \item{Pipeta graduada a 5 ml}
    \item{Pinza para crisol}
    \item{Crisol de porcelana de 100 ml}
    \item{Balanza Analítica}  
    \item{Mufla}
\end{itemize}

\subsection{Reactivos} 
\begin{itemize}
    \item{Ácido sulfúrico Concentrado} 
\end{itemize}

\subsection{Datos experimentales} 
\begin{itemize}
    \item{\textbf{Pesos:}} \\
    $m_c{}_1$ = 23,3289g \\ 
    $m_c{}_2$ = 23,3282g \\
    $m_c{}_p$ = 23,3286g \\
    $m_t{}_1$ = 24,0574g \\
    $m_t{}_2$ = 23,3936g \\
    $m_t{}_3$ = 23,3934g \\ 
    $m_m$ = 5,000g \\ 

    \item{\textbf{Volúmenes:}} \\
    $V_{H2SO4}$ = 0,5 mL 

    \item{\textbf{Donde:}} \\ 
    $m_c$ : Pesadas sucesivas del crisol vació \\ 
    $m_c{}_p$ : Peso promedio del crisol vació \\ 
    $m_t{}$ : Pesadas sucesivas del crisol con la muestra \\
    $m_m$ : Peso de la muestra \\ 
    $V_{H2SO4}$ : Volumen de ácido sulfúrico
\end{itemize}

\subsection{Cálculos y resultados} 
Porcentaje de cenizas sulfatadas = $100*(\frac{23,3934-23,32856}{5,000})$\\
Porcentaje de cenizas sulfatadas = $1,296\%$

\subsection{Discusión} 
El porcentaje obtenido de cenizas sulfatadas fue de 1,296\%, correspondientes a azúcar orgánica, sobrepasando lo establecido en los límites de la azúcar refinada por la norma COVENIN 234:1995 Azúcar Refinado. 



\newpage

\section{Polarización del azúcar}

\subsection{Fundamento} 
Se basa en la medición de la propiedad que tienen las soluciones de sacarosa, para hacer girar el plano de polarización de un rayo de luz, siendo este giro proporcional a la cantidad de sacarosa presente en la solución. 

\subsection{Materiales} 
\begin{itemize}
    \item{Matraz de 100 ml}
    \item{Vaso de precipitado de 400 mm}
    \item{Papel de filtro (filtración rápida)}
    \item{Termómetro de 0 a 100ºC}
    \item{Espátula}
    \item{Vaso de precipitado de 150 ml}  
    \item{Sacarímetro graduado}
    \item{Balanza Analítica} 
    \item{Tubo de polarización de 200 mm ± 0,03 mm de longitud } 
\end{itemize}

\subsection{Reactivos} 
\begin{itemize}
    \item{Acetato Básico de plomo} 
\end{itemize}

\subsection{Datos experimentales} 
La determinación de polarización del azúcar no se logró realizar por falta de tiempo en el laboratorio. 



\newpage

\section{Conclusión}
En esta practica se hallo porcentaje de cenizas sulfatadas mediante una incineración simple luego de la sulfatación de la muestra de azúcar, así realizo de manera cualitativa la determinación de azucares reductores. Con las normas COVENIN 234:1995 Azúcar Refinado y NMX-F-496-SCFI-2011 Determinacion de reductores totales en azucares. Se contrastaron los resultados, donde se discutieron las diferencias y posibles causas entre los resultados obtenidos y los establecidos en las normas consultadas.

\newpage

\section {Bibliografías}
\begin{itemize}
    \item {COVENIN 234: 1995. AZUCAR REFINADO}
    \item {COVENIN 3107:1994. AZUCAR. DETERMINACION DE AZUCARES REDUCTORES}
    \item {NMX-F-496-SCFI-2011 DETERMINACION DE REDUCTORES TOTALES EN AZUCARES}
    \item{NMX-F-082-SCFI-2012 INDUSTRIA AZUCARERA Y ALCOHOLERA - CENIZAS SULFATADAS EN AZÚCARES - MÉTODO GRAVIMÉTRICO }
\end{itemize}


\printbibliography[
heading=bibintoc,
title={Bibliografías}
]


\end{document}
